\documentclass[letter,12pt]{scrartcl}
%\usepackage[margin=0.5in]{geometry}

\begin{document}
	\title{ ALBI-345 - Exam 2}
	\date{Ocober 27, 2021}
	\maketitle
	Name:
	\vspace{0.25in}\hrule
	\begin{enumerate}
		
%		<!--
%		To convet to PDF:
%		
%		pandoc Exam2-2021.md \
%		-N \
%		-V documentclass:scrartcl \
%		-V geometry:margin=1in \
%		-V fontsize=14pt \
%		-V \setkomafont{chapter}{\normalfont\huge\sffamily\bfseries\color{blue}}
%		-o Exam2-2021.pdf
%		
%		Some LaTeX code, such as \newline, \hrule, and \pagebreak can be used to help
%		format PDFs
%		-->
		
%		**Chemical ecology**
%		
%		* Insects rely heavily on semiochemicals to get information from the environment.
%		* Insects use pheromones for intraspecific communication. There are many types of pheromones: sex pheromones, aggregation pheromones, alarm pheromones, trail-following pheromones, etc.
%		* Synthetic pheromones extensively in insect pest control for monitoring, population reduction (attracticides), and mating disruption.
%		* Insects use kairomones for finding food: host plants, nectar, prey and host insects.
%		-->
		
		\item State two ways in which pheromones are being used in modern pest control. (2 points)
		\vspace{1in}\hrule
		
		\item What is the difference between pheromones and kairomones? (1 point)
		\vspace{1in}\hrule
		
%		<!--
%		**Mimicry and Camouflage**
%		
%		* Batesian mimicry: a non-harmful species (mimic) has evolved to look like a harmful species (model)
%		* Mullerian mimicry: two or more unrelated noxious, or dangerous, organisms exhibit closely similar warning coloration (usually patterns often yellow, red and black stripes)
%		* Camouflage
%		-->
		
		\item There are many species of flies that mimic bees and wasps. What type of mimicry is this and how does it benefit the flies? (2 points)
		\vspace{1in}\hrule
		
		\item Insect camouflage may not work unless an insect behaves appropriately. Explain. (1 point)
		\vspace{1in}\hrule
		
%		<!--
%		**Population dynamics**
%		
%		* Exponential model
%		* Logistic model
%		* Carrying capacity (k): maximum population size that can be sustained by the environment
%		* r-selected and k-selected species
%		* Lotka-Voltera model for population dynamics of interacting species
%		-->
		
		\item Most insect species are referred to as having an “r-selected reproductive strategy”. What does this mean? (1 point)
		\vspace{1in}\hrule
		
		\item The parameter K in the logistic model, also known as the carrying capacity (pick 1 of the following; 1 point):
		\begin{itemize}
			\item is the maximum number of individuals that can be sustained by available resources
			\item is the total amount of food and other resources available to the population
			\item is maximum population growth rate
			\item all of the above			
		\end{itemize}
		\bigskip\hrule
		
%		<!--
%		**Invasive species**
%		
%		* An invasive species is alien and causes harm.
%		* Invasive species readily establish on tropical islands because they have "escaped from natural enemies", weather is benign (no winter, high humidity), and there is usually plenty of food and water.
%		* Fifty percent of Guam's invasive insects belong to Order Hemiptera.
%		-->
		
		\item What two criteria must be present for an organism to be considered an “invasive species”? (2 points)
		\vspace{1in}\hrule
		
		\item Tropical islands are much more susceptible to damage from invasive species than are continents? Give two reasons for this difference in risk. (2 points)
		\vspace{1in}\hrule
		
		\item About half of invasive insect species arriving on Guam in recent years belong to one order. What is the name of this order and why is this order such a big problem for Guam's biosecurity? (2points)
		\vspace{1in}\hrule
		
		\pagebreak
		
		\item Give the scientific names for 3 important invasive insect species on Guam and briefly describe the damage they are causing. (6 points)
		\vspace{1in}\hrule
		
%		<!--
%		**Applied entomology**
%		
%		* Pest control based solely on pesticide application is limited because of pesticide resistance and nontarget effects.
%		* Integrated pest management (IPM): one or more pest control tactics are selected based on knowledge of the current situation gained from pest monitoring.
%		* IPM tactics include pesticide application, sanitation, biological control, physical control, cultural control, and "do nothing".
%		* Economic threshold level (EIL): the population density at which pest control should be initiated to prevent economic damage.
%		* If possible, broad spectrum pesticides should be avoided because they may kill beneficial insects which will lead to an uncontrolled resurgence of pests.
%		-->
		
		\item Sometimes, after a farmer applies a broad-spectrum insecticide, there is a pest population explosion referred to as “pest resurgence”? Explain what might be happening here. (2 points)
		\vspace{1in}\hrule
		
		\item Give a brief definition of integrated pest management (IPM). (2 points)
		\vspace{1in}\hrule
		
		\item Which of the following are valid IPM tactics? (make 1 choice; 1 point)
		\begin{itemize}
			\item pesticide application
			\item release of biological control agents
			\item sanitation
			\item crop rotation
			\item all of the above
		\end{itemize}
		\bigskip\hrule
				
		\item Insect herbivores are sometimes used as biological control agents in weed management programs. Briefly discuss the risks involved and how these risks can be minimized. (2 points)
		\vspace{1in}\hrule
		
		\newpage
		
		\item Classify each insect specimen to Order. (1 point awarded for each correctly identified)\\
		1.\newline\hrule
		2.\newline\hrule
		3.\newline\hrule
		4.\newline\hrule
		5.\newline\hrule
		6.\newline\hrule
		7.\newline\hrule
		8.\newline\hrule
		9.\newline\hrule
		10.\newline\hrule
		
	\end{enumerate}
\end{document}